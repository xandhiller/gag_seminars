\documentclass{article}
\usepackage{amsmath}      % Mathematics
\usepackage{amssymb}      % Mathematics
\usepackage{listings}     % Listings
%\usepackage{esint}       % Mathematics (Causing problems with mdframed)
\usepackage{color}        % Listings
\usepackage{courier}      % Listings
\usepackage[oldvoltagedirection]{circuitikz}   % Circuits
\usepackage{titlesec}     % Section Formatting
\usepackage{stmaryrd}     % \mapsfrom arrow. 
\usepackage{mathtools}    % \coloneqq
\usepackage{svg}
\usepackage{import}
\usepackage{pdfpages}
\usepackage{transparent}
\usepackage{xcolor}
\usepackage{blindtext}
\usepackage[hidelinks]{hyperref}
\usepackage{tabularx}
\usepackage{mdframed}
%%%%%%%%%%%%%%%%%%%%%%%%%%%%%%%%%%%%%%%%%%%%%%%%%%%%%%%%%%%%%%%%%%%%%%%%%%%%%%%%
% Math Macros
%%%%%%%%%%%%%%%%%%%%%%%%%%%%%%%%%%%%%%%%%%%%%%%%%%%%%%%%%%%%%%%%%%%%%%%%%%%%%%%%
% Standard Notation for Vectors in Computer Vision
\usepackage{mdframed}

% Footnote preferences
\renewcommand{\thefootnote}{\fnsymbol{footnote}}
  % Listings Prerequisites
%%%%%%%%%%%%%%%%%%%%%%%%%%%%%%%%%%%%%%%%%%%%%%%%%%%%%%%%%%%%%%%%%%%%%%%%%%%%%%%%
\definecolor{codegreen}{rgb}{0,0.6,0}
\definecolor{codegray}{rgb}{0.5,0.5,0.5}
\definecolor{codepurple}{rgb}{0.58,0,0.82}
\definecolor{backcolour}{rgb}{0.87,0.87,0.87}
\lstdefinestyle{mystyle}{
  backgroundcolor=\color{backcolour},   
  commentstyle=\color{codegreen},
  keywordstyle=\color{magenta},
  numberstyle=\tiny\color{codegray},
  stringstyle=\color{codepurple},
  basicstyle=\footnotesize\ttfamily,
  breakatwhitespace=false,         
  breaklines=true,                 
  captionpos=b,                    
  keepspaces=true,                 
  %numbers=left,                    
  numbersep=5pt,                  
  showspaces=false,                
  showstringspaces=false,
  showtabs=false,                  
  tabsize=2
}
\lstset{style=mystyle} 

% Allows you to refer to the author and title in text
\makeatletter
\let\inserttitle\@title
\makeatother
\makeatletter
\let\insertauthor\@author
\makeatother

%\newcommand{\qed}{$\square$}
\newcommand{\QED}{\hspace*{\fill}$\square$}
\newcommand{\iddots}{\reflectbox{$\ddots$}}
%----------------------%
% Counter for Custom Environments
%----------------------% 
%\rtask{label}
%\section{Task \ref{label}. Blah}
%----------------------%
% Custom Environments
%----------------------% 
\newcounter{question} 
\newcommand{\qanda}[2]{\AtEndDocument{\stepcounter{question}Q\arabic{question}. \textit{#1} \vspace{2mm}\newline \AtEndDocument{A\arabic{question}. #2 \newline \newline}}}
%----------------------%

/home/alex/Meta/templates/tex/dtper_article.tex
%%%%%%%%%%%%%%%%%%%%%%%%%%%%%%%%%%% FORMATTING %%%%%%%%%%%%%%%%%%%%%%%%%%%%%%%%%
\author{Tim Gou}
\title{\textbf{Groups, Analysis, and Geometry Seminars}: Harmonic Analysis of $\mathcal{SU}(2)$}
\date{20th August, 2020}
\usepackage{geometry} % Formatting
%%%%%%%%%%%%%%%%%%%%%%%%%%%%%%%%%%%%%%%%%%%%%%%%%%%%%%%%%%%%%%%%%%%%%%%%%%%%%%%%

\begin{document}

\maketitle

\section{Introduction} 

Suppose $f$ is $2\pi$-periodic, complex valued, integrable over $[0, 2\pi)$, then
\begin{equation}
    \widehat{f}(n) = \frac{1}{2\pi} \int^{2\pi}_{0} f(x) e^{-inx} \ dx
\end{equation}
with $n \in \mathbb{Z}$, is the Fourier transform of $f$. Question: What is $e^{inx}$? Why is $n \in \mathbb{Z}$? Answer: $e^{inx}$ is the character of circle group, denoted by $\mathbb{T}$ (i.e. $e^{ix} \in \mathbb{T}$).

A character is a continuous homomorphism from a locally compact Abelian group $G$ to $\mathbb{T}$: $ \chi : G \rightarrow \mathbb{T} $
where
\begin{equation}
    \chi(gh) = \chi(g)\chi(h)
\end{equation}
for $g,h \in G$. Let's work out $\chi : \mathbb{R} \rightarrow \mathbb{T}$ first, where $\mathbb{R}$ is the group $(\mathbb{R}, +)$. Since $\chi(0)=1$ (identity to identity) and $\chi$ is continuous, then $\exists a > 0$ such that
\begin{equation}
    \int^{a}_{0} \chi(y) \ dy.
\end{equation}
Let $\xi = \int^{a}_{0} \chi(y) \ dy$, then 
\begin{equation}
    \chi(x)\xi = \int^{a}_{0} \chi(x+y) \ dy = \int^{a+x}_{x} \chi(t) \ dt
\end{equation} 
so
\begin{equation}
    \chi(x) = \xi^{-1} \int^{a+x}_{x} \chi(t) \ dt
\end{equation}
and
\begin{equation}
    \begin{split}
        \chi'(x) &= \xi^{-1} \left(\chi(a+x) - \chi(x) \right) \\
                 &= \xi^{-1} \chi(x) \left(\chi(a) - 1\right) \\
                 &= c \chi(x).
    \end{split}
\end{equation}
We have an ODE
\begin{equation}
    \chi'(x) = c \chi(x)
\end{equation}
where solving the equation gives us
\begin{equation}
    \chi(x) = e^{cx}.
\end{equation}
Solving it gives us $\chi(x) = e^{cx}$. Since $| \chi | = 1$, then $c=i\lambda$ with $\lambda \in \mathbb{R}$. Thus $\chi(x) = e^{i\lambda x}$ and we have characters of $\mathbb{R}$, all the $\chi_{\lambda}$ form a dual group of $\mathbb{R}$, denoted by $\widehat{\mathbb{R}}$. 

Since we identify each $\chi_{\lambda}$ with $\lambda \in \mathbb{R}$, then 
\begin{equation}
    \widehat{\mathbb{R}} \cong \mathbb{R}.
\end{equation}

To work out $\widehat{\mathbb{T}}$, notice that 
\begin{equation}
    \mathbb{T} \cong \mathbb{R} / 2\pi \mathbb{Z}
\end{equation}
i.e. each element in $[0, 2\pi)$ is a representative of the cosets of $\mathbb{R} / 2 \pi \mathbb{Z}$. Suppose $x, y \in \mathbb{R}/2\pi \mathbb{Z}$ and $x+y = 2\pi$, then
\begin{equation}
    \chi(x+y) = \chi(0) = 1 = e^{i\lambda (x+y)}
\end{equation}
we know $\lambda \in \mathbb{R}$, but the only way $e^{i\lambda(x+y)}=1$ is that $\lambda \in \mathbb{Z}$. So all the $\chi_{n}(x) = e^{inx}$ for the dual group $\widehat{\mathbb{T}}$, and 
\begin{equation}
    \widehat{\mathbb{T}} \cong \mathbb{Z}.
\end{equation}
Similarly, we have $\mathbb{R}^{n} \cong \mathbb{R}^{n}$ and $\widehat{\mathbb{T}} \cong \mathbb{Z}^{n}$.

\begin{theorem}
    If $G$ is compact, $\widehat{G}$ is discrete.
\end{theorem} 

In addition, $\{ e^{inx} : n \in \mathbb{Z} \}$ form an orthonormal basis for the Hilbert space $L^{2}(\mathbb{T})$, with respect to its inner product, i.e.
\begin{equation}
    \begin{split}
        \langle e^{imx}, e^{inx} \rangle 
        &= \frac{1}{2\pi} \int^{2\pi}_{0} e^{imx} e^{-inx} \ dx \\
        &= \delta_{mn} = 
        \begin{cases}
            \case 1, \ m=n \\
            \case 0, \ m\neq n \\
        \end{cases}
    \end{split}
\end{equation}

If $G$ is non-Abelian, the analogy generalising the characters is called the \textit{irreducible unitary representation} of $G$:
\begin{equation}
    \sigma : G \rightarrow U(\mathcal{H})
\end{equation}
for $\mathcal{H}$, some Hilbert space. 
\begin{equation}
    \begin{split}
        \sigma(gh) &= \sigma(g)\sigma(h) \\
        \sigma(e) &= \sigma(g g^{-1}) = \sigma(g)\sigma(g^{-1}) = \sigma(g)\sigma(g)^{*} \\
        \langle \sigma(g)u, \sigma(g)v \rangle &= \langle u,v \rangle \\
    \end{split}
\end{equation}
for $u,v \in \mathcal{H}$. We'll look at the irreducible representations of $\mathcal{SU}(2)$. $\mathcal{SU}(2)$ is the first compacy and non-abelian group we normally look at in Harmonic Analysis. 

\section{Aspects of $\mathcal{SU}(2)$} 

We begin by looking at $\mathcal{U}(2)$, a group of unitary transformations of $\mathbb{C}^{2}$. 
\begin{equation}
    A A^{*} = A^{*}A = I, A \in \mathcal{U}(2)
\end{equation}
$\mathcal{SU}(2) \subset \mathcal{U}(2)$ where elements of $\mathcal{SU}(2)$ have determinant 1. 
\begin{equation}
    \begin{cases}
        \case \det\left( AB \right) = \det\left( A \right) \det\left( B \right), \ A,B \in \mathcal{SU}(2) \\
        \case \det\left( I \right) = \det\left( A A^{-1} \right) = \det\left( A \right) \det\left( A^{-1} \right)
    \end{cases}
\end{equation}
suppose 
\begin{equation}
    \begin{pmatrix}
        \alpha & \beta \\
        \gamma & \eta \\
    \end{pmatrix} \in \mathcal{U}(2)
\end{equation}
then, $| \alpha |^{2} + | \beta |^{2} = | \gamma |^{2} + | \eta |^{2} = 1$ and $\alpha \overline{\gamma} + \beta \overline{eta} = 0$. So,
\begin{equation}
    \begin{pmatrix}
        \alpha \\ \beta \\
    \end{pmatrix},
    \begin{pmatrix}
        \gamma \\ \eta \\
    \end{pmatrix}
\end{equation}
are unit vectors and orthogonal. Thus, 
\begin{equation}
    \begin{pmatrix}
        \gamma \\ \eta \\
    \end{pmatrix}
    = e^{i\theta}
    \begin{pmatrix}
        -\overline{\beta} \\ \overline{\alpha} \\
    \end{pmatrix},
    \ \theta \in \mathbb{R}.
\end{equation}
Hence we have
\begin{equation}
    A =
    \begin{pmatrix}
        \alpha & \beta \\
        -e^{i\theta}\overline{\beta} & e^{i\theta}\overline{\alpha} \\
    \end{pmatrix}
    \ \in \mathcal{U}(2)
\end{equation}
with $\text{det}\left( A \right) = e^{i\theta}$ and $\text{det}\left( A^{*} \right) = e^{-i\theta}$.
If $A \in \mathcal{SU}(2)$, then 
\begin{equation}
    A = 
    \begin{pmatrix}
        \alpha & \beta \\
        -\overline{\beta} & \overline{\alpha}
    \end{pmatrix}
    \in \mathcal{SU}(2).
\end{equation}

\end{document}

