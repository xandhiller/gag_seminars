\documentclass{article}
\usepackage{amsmath}      % Mathematics
\usepackage{amssymb}      % Mathematics
\usepackage{listings}     % Listings
%\usepackage{esint}       % Mathematics (Causing problems with mdframed)
\usepackage{color}        % Listings
\usepackage{courier}      % Listings
\usepackage[oldvoltagedirection]{circuitikz}   % Circuits
\usepackage{titlesec}     % Section Formatting
\usepackage{stmaryrd}     % \mapsfrom arrow. 
\usepackage{mathtools}    % \coloneqq
\usepackage{svg}
\usepackage{import}
\usepackage{pdfpages}
\usepackage{transparent}
\usepackage{xcolor}
\usepackage{blindtext}
\usepackage[hidelinks]{hyperref}
\usepackage{tabularx}
\usepackage{mdframed}
\input{/home/alex/Meta/templates/tex/tex_macros}
/home/alex/Meta/templates/tex/dtper_article.tex
%%%%%%%%%%%%%%%%%%%%%%%%%%%%%%%%%%% FORMATTING %%%%%%%%%%%%%%%%%%%%%%%%%%%%%%%%%
\author{Tim Gou}
\title{\textbf{Groups, Analysis, and Geometry Seminars}: Harmonic Analysis of $\mathcal{SU}(2)$}
\date{20th August, 2020}
\usepackage{geometry} % Formatting
%%%%%%%%%%%%%%%%%%%%%%%%%%%%%%%%%%%%%%%%%%%%%%%%%%%%%%%%%%%%%%%%%%%%%%%%%%%%%%%%

\begin{document}

\maketitle

\section{Introduction} 

Suppose $f$ is $2\pi$-periodic, complex valued, integrable over $[0, 2\pi)$, then
\begin{equation}
    \widehat{f}(n) = \frac{1}{2\pi} \int^{2\pi}_{0} f(x) e^{-inx} \ dx
\end{equation}
with $n \in \mathbb{Z}$, is the Fourier transform of $f$. Question: What is $e^{inx}$? Why is $n \in \mathbb{Z}$? Answer: $e^{inx}$ is the character of circle group, denoted by $\mathbb{T}$ (i.e. $e^{ix} \in \mathbb{T}$).

A character is a continuous homomorphism from a locally compact Abelian group $G$ to $\mathbb{T}$: $ \chi : G \rightarrow \mathbb{T} $
where
\begin{equation}
    \chi(gh) = \chi(g)\chi(h)
\end{equation}
for $g,h \in G$. Let's work out $\chi : \mathbb{R} \rightarrow \mathbb{T}$ first, where $\mathbb{R}$ is the group $(\mathbb{R}, +)$. Since $\chi(0)=1$ (identity to identity) and $\chi$ is continuous, then $\exists a > 0$ such that
\begin{equation}
    \int^{a}_{0} \chi(y) \ dy.
\end{equation}
Let $\xi = \int^{a}_{0} \chi(y) \ dy$, then 
\begin{equation}
    \chi(x)\xi = \int^{a}_{0} \chi(x+y) \ dy = \int^{a+x}_{x} \chi(t) \ dt
\end{equation} 
so
\begin{equation}
    \chi(x) = \xi^{-1} \int^{a+x}_{x} \chi(t) \ dt
\end{equation}
and
\begin{equation}
    \begin{split}
        \chi'(x) &= \xi^{-1} \left(\chi(a+x) - \chi(x) \right) \\
                 &= \xi^{-1} \chi(x) \left(\chi(a) - 1\right) \\
                 &= c \chi(x).
    \end{split}
\end{equation}
We have an ODE
\begin{equation}
    \chi'(x) = c \chi(x)
\end{equation}
where solving the equation gives us
\begin{equation}
    \chi(x) = e^{cx}.
\end{equation}
Solving it gives us $\chi(x) = e^{cx}$. Since $| \chi | = 1$, then $c=i\lambda$ with $\lambda \in \mathbb{R}$. Thus $\chi(x) = e^{i\lambda x}$ and we have characters of $\mathbb{R}$, all the $\chi_{\lambda}$ form a dual group of $\mathbb{R}$, denoted by $\widehat{\mathbb{R}}$. 

Since we identify each $\chi_{\lambda}$ with $\lambda \in \mathbb{R}$, then 
\begin{equation}
    \widehat{\mathbb{R}} \cong \mathbb{R}.
\end{equation}

To work out $\widehat{\mathbb{T}}$, notice that 
\begin{equation}
    \mathbb{T} \cong \mathbb{R} / 2\pi \mathbb{Z}
\end{equation}
i.e. each element in $[0, 2\pi)$ is a representative of the cosets of $\mathbb{R} / 2 \pi \mathbb{Z}$. Suppose $x, y \in \mathbb{R}/2\pi \mathbb{Z}$ and $x+y = 2\pi$, then
\begin{equation}
    \chi(x+y) = \chi(0) = 1 = e^{i\lambda (x+y)}
\end{equation}
we know $\lambda \in \mathbb{R}$, but the only way $e^{i\lambda(x+y)}=1$ is that $\lambda \in \mathbb{Z}$. So all the $\chi_{n}(x) = e^{inx}$ for the dual group $\widehat{\mathbb{T}}$, and 
\begin{equation}
    \widehat{\mathbb{T}} \cong \mathbb{Z}.
\end{equation}
Similarly, we have $\mathbb{R}^{n} \cong \mathbb{R}^{n}$ and $\widehat{\mathbb{T}} \cong \mathbb{Z}^{n}$.

\begin{theorem}
    If $G$ is compact, $\widehat{G}$ is discrete.
\end{theorem} 

In addition, $\{ e^{inx} : n \in \mathbb{Z} \}$ form an orthonormal basis for the Hilbert space $L^{2}(\mathbb{T})$, with respect to its inner product, i.e.
\begin{equation}
    \begin{split}
        \langle e^{imx}, e^{inx} \rangle 
        &= \frac{1}{2\pi} \int^{2\pi}_{0} e^{imx} e^{-inx} \ dx \\
        &= \delta_{mn} = 
        \begin{cases}
            \case 1, \ m=n \\
            \case 0, \ m\neq n \\
        \end{cases}
    \end{split}
\end{equation}

If $G$ is non-Abelian, the analogy generalising the characters is called the \textit{irreducible unitary representation} of $G$:
\begin{equation}
    \sigma : G \rightarrow U(\mathcal{H})
\end{equation}
for $\mathcal{H}$, some Hilbert space. 
\begin{equation}
    \begin{split}
        \sigma(gh) &= \sigma(g)\sigma(h) \\
        \sigma(e) &= \sigma(g g^{-1}) = \sigma(g)\sigma(g^{-1}) = \sigma(g)\sigma(g)^{*} \\
        \langle \sigma(g)u, \sigma(g)v \rangle &= \langle u,v \rangle \\
    \end{split}
\end{equation}
for $u,v \in \mathcal{H}$. We'll look at the irreducible representations of $\mathcal{SU}(2)$. $\mathcal{SU}(2)$ is the first compacy and non-abelian group we normally look at in Harmonic Analysis. 

\section{Aspects of $\mathcal{SU}(2)$} 

We begin by looking at $\mathcal{U}(2)$, a group of unitary transformations of $\mathbb{C}^{2}$. 
\begin{equation}
    A A^{*} = A^{*}A = I, A \in \mathcal{U}(2)
\end{equation}
$\mathcal{SU}(2) \subset \mathcal{U}(2)$ where elements of $\mathcal{SU}(2)$ have determinant 1. 
\begin{equation}
    \begin{cases}
        \case \det\left( AB \right) = \det\left( A \right) \det\left( B \right), \ A,B \in \mathcal{SU}(2) \\
        \case \det\left( I \right) = \det\left( A A^{-1} \right) = \det\left( A \right) \det\left( A^{-1} \right)
    \end{cases}
\end{equation}
suppose 
\begin{equation}
    \begin{pmatrix}
        \alpha & \beta \\
        \gamma & \eta \\
    \end{pmatrix} \in \mathcal{U}(2)
\end{equation}
then, $| \alpha |^{2} + | \beta |^{2} = | \gamma |^{2} + | \eta |^{2} = 1$ and $\alpha \overline{\gamma} + \beta \overline{\eta} = 0$. So,
\begin{equation}
    \begin{pmatrix}
        \alpha \\ \beta \\
    \end{pmatrix},
    \begin{pmatrix}
        \gamma \\ \eta \\
    \end{pmatrix}
\end{equation}
are unit vectors and orthogonal. Thus, 
\begin{equation}
    \begin{pmatrix}
        \gamma \\ \eta \\
    \end{pmatrix}
    = e^{i\theta}
    \begin{pmatrix}
        -\overline{\beta} \\ \overline{\alpha} \\
    \end{pmatrix},
    \ \theta \in \mathbb{R}.
\end{equation}
Hence we have
\begin{equation}
    A =
    \begin{pmatrix}
        \alpha & \beta \\
        -e^{i\theta}\overline{\beta} & e^{i\theta}\overline{\alpha} \\
    \end{pmatrix}
    \ \in \mathcal{U}(2)
\end{equation}
with $\text{det}\left( A \right) = e^{i\theta}$ and $\text{det}\left( A^{*} \right) = e^{-i\theta}$.
If $A \in \mathcal{SU}(2)$, then 
\begin{equation}
    A = 
    \begin{pmatrix}
        \alpha & \beta \\
        -\overline{\beta} & \overline{\alpha}
    \end{pmatrix}
    \in \mathcal{SU}(2).
\end{equation}
\begin{equation}
    \mathcal{SU}(2) = 
    \left\{ 
        \begin{pmatrix}
            \alpha & \beta \\
            -\overline{\beta} & \overline{\alpha}
        \end{pmatrix}
        :
        | \alpha |^{2} + | \beta |^{2} = 1
    \right\}
\end{equation}
where $\alpha, \beta \in \mathbb{C}$. If $\alpha = a+ib$ and $\beta = c+id$ then
\begin{equation}
    \begin{split}
        | \alpha |^{2} + | \beta |^{2} &= 1 \\
        \left| \sqrt{a^{2} + b^{2}} \right|^{2} + \left| \sqrt{c^{2} + d^{2}} \right|^{2} &= 1 \\
        {a^{2} + b^{2}} + {c^{2} + d^{2}} &= 1 \\
    \end{split}
\end{equation}
and so $\mathcal{SU}(2) \cong S^{3}$, the $3-$sphere. $\mathbb{T} \subset \mathcal{SU}(2)$
\begin{equation}
    \mathbb{T} = \left\{ 
        \begin{pmatrix}
            \alpha & 0 \\
            0 & \overline{\alpha}
        \end{pmatrix}
        :
        | \alpha | = 1
    \right\}
\end{equation}

\begin{theorem}
    Given $A \in \mathcal{SU}(2)$, $\exists V \in \mathcal{SU}(2)$, such that 
    \begin{equation}
        V A V^{-1} = T
    \end{equation} 
    for some $T \in \mathbb{T}$.
\end{theorem}

\begin{proof}
    If $A \in \mathcal{U}(2)$, $A$ is normal, then by the \textit{spectral theorem}, $\exists V \in \mathcal{U}(2)$ such that
    \begin{equation}
        VA v^{-1} = 
        \begin{pmatrix}
            \alpha & 0 \\
            0 & \beta \\
        \end{pmatrix}
    \end{equation}
    with $| \alpha | = | \beta | = 1$.
\end{proof}

If $A \in \mathcal{SU}(2)$, then $\beta = \overline{\alpha}$ . So, $V \in \mathcal{U}(2)$ , $A \in \mathcal{SU}(2)$, we have
\[
    VAV^{-1} = 
    \begin{pmatrix}
        \alpha & 0 \\
        0 & \overline{\alpha}
    \end{pmatrix}.
\]
To make $V$ special unitary, let $\tilde{V} = \det\left( V \right)^{1/2}V$ such that $\det\left( \tilde{V} \right) = 1$ and we have $\tilde{V}A\tilde{V}^{-1}=T$.

$\mathcal{SU}(2)$ is a compact, connected, Lie group, (?) we have a theorem to generalise this conjugation relation.

\begin{theorem}
    If $G$ is a compact, connected, Lie group, $\mathbb{T}$ is its maximal torus, then for $\chi \in G$, $\exists g \in G$, such that
    \begin{equation}
        g \chi g^{-1} = t \quad : \quad t \in \mathbb{T}.
    \end{equation}
\end{theorem}

\begin{theorem}
    (proposition?) Define $O_{x} = \{ g \chi g^{-1} : g \in G \}$, then for $\mathcal{SU}(2)$, every $O_{x}$ intersects $\mathbb{T}$ at exactly two points.
\end{theorem}

\begin{proof}
    We have $g \chi g^{-1} = t$ , let
    \[
        g_1 = 
        \begin{pmatrix}
            0 & 1 \\
            -1 & 0 \\
        \end{pmatrix}
        \Rightarrow
        g^{-1} = 
        \begin{pmatrix}
            0 & -1 \\
            1 & 0 \\
        \end{pmatrix}
    \]
    so 
    \[
        \begin{split}
            (g_1 g) \chi (g_1 g)^{-1} &= 
            \begin{pmatrix}
                0 & 1 \\
                -1 & 0 \\
            \end{pmatrix}
            \begin{pmatrix}
                e^{i\theta} & 0 \\
                0 & e^{-i\theta} \\
            \end{pmatrix}
            \begin{pmatrix}
                0 & -1 \\
                1 & 0 \\
            \end{pmatrix} \\
                                      &=
            \begin{pmatrix}
                e^{-i\theta} & 0 \\
                0 & e^{i\theta}
            \end{pmatrix}
        \end{split}
    \]
\end{proof}

Now, we describe the irreducible representations of $\mathcal{SU}(2)$. $\mathcal{SU}(2)$ naturally acts on $\mathbb{C}^{2}$ ($\sigma$ is just the identity mapping) i.e.
\begin{equation}
    \begin{pmatrix}
        \alpha & \beta \\
        -\overline{\beta} & \overline{\alpha} \\
    \end{pmatrix}
    \begin{pmatrix}
        z \\ w
    \end{pmatrix} 
    = 
    \begin{pmatrix}
        \alpha z + \beta w \\
        - \overline{\beta}z + \overline{\alpha}w
    \end{pmatrix}
\end{equation}

We are interested in $\mathcal{SU}(2)$ acting on the vector spaces which are built out of $\mathbb{C}^{2}$, e.g.
\begin{equation}
    L^{2}(\mathcal{SU}(2)) \cong L^{2}(S^{3})
\end{equation}
made up of functions
\begin{equation}
    f : \mathbb{C}^{2} \rightarrow \mathbb{C}
\end{equation}
and let $P$ be the space of all polynomials with two complex variables, i.e. $f \in P$ such that
\begin{equation}
    f(z,w) = \sum_{j,k}^{} a_{jk} z^{j} w^{k}
\end{equation}
and 
\begin{equation}
    P \subset L^{2}(S^{3}) 
\end{equation}
where both spaces are infinite dimensional. But $\mathcal{SU}(2)$ is compact, so $d_{\sigma} < \infty$. This means we want to find subspaces of $P$ which are:
\begin{itemize}
    \item finite dimensional
    \item invariant under the action of $\mathcal{SU}(2)$
\end{itemize}

Thus, we pick $P_m$ , the space of homogeneous polynomials of degree $m$. For $f \in P_m$ 
\begin{equation}
    f(z,w) = \sum_{j=0}^{m} a_j z^{m-j} w^{j}
\end{equation}
and let 
\begin{equation}
    A_{\alpha, \beta} := 
    \begin{pmatrix}
        \alpha & \beta \\
        -\overline{\beta} & \overline{\alpha}
    \end{pmatrix},
\end{equation}
we define the regular representation of $\sigma(A_{\alpha,\beta})$ on $f$:
\begin{equation}
    \begin{split}
        \sigma(A_{\alpha, \beta})f(z,w) &= f(A_{\alpha,\beta}(z,w))\\
                                        &= f(\overline{\alpha} z - \beta w, 
                                             \overline{\beta}z + \alpha w)
    \end{split}
\end{equation}

\begin{theorem}
    $P_m$ is invariant under $\sigma(A_{\alpha,\beta})$.
\end{theorem}

\begin{proof}
    Suppose monomials $z^{m-j}w^{j} \in P_m$ then 
    \begin{equation}
        f(\overline{\alpha}z - \beta w, \overline{\beta}z + \alpha w)
        =
        (\overline{\alpha}z - \beta w)^{m-j} (\overline{\beta}z + \alpha w)^{j} \in P_m.
    \end{equation}
    So, $P_m$ is invariant under $\sigma$.
\end{proof}
Now, we want to define an inner product, i.e.
\begin{equation}
    \langle f,g \rangle = \int_{S^{3}} f \overline{g} \ d\mu
\end{equation}
where $\mu(S^{3})=1$ is the surface measure of $S^{3}$ which is normalised to 1.

So, we'll show monomials $z^{m-j}w^{j}$ are orthogonal in $P$ with respect to this inner product. To achieve this, we use polar coordinates.

Suppose $Z = (z,w) \in C^{2}$ and $Z = r \tilde{z}$, where $\tilde{z} \in S^{3}$, and $r = | z | = \sqrt{| z |^{2} + | w |^{2}}$. We define Lebesgue measure on $\mathbb{C}^{2}$ as 
\begin{equation}
    \begin{split}
        dz &= dz \ dw \\
           &= d_a \ d_b \ d_c \ d_{d} \\
           &= r^{3} \ dr \ d\tilde{\mu}(\tilde{z}) \ : \ \text{(un-normalised)} \\
           &= 2 \pi^{2} r^{3} \ dr \ d\mu (\tilde{z}) \ : \ \text{(normalised)}
    \end{split}.
\end{equation}
To see this, we have $\varphi_1, \varphi_2 \in [0, \pi]$, $\theta \in [0, 2\pi)$. 
\begin{equation}
    \begin{split}
        a &= r \cos\left(\varphi_1\right) \\
        b &= r \sin\left(\varphi_1\right) \sin\left(\varphi_2\right) \\
        c &= r \sin\left(\varphi_1\right) \sin\left(\varphi_2\right) \cos\left(\theta\right) \\
        d &= r \sin\left(\varphi_1\right) \sin\left(\varphi_2\right) \sin\left(\theta\right) .\\
    \end{split}
\end{equation}
The Jacobian matrix is then
\begin{equation}
    J = 
    \begin{bmatrix} 
        \frac{\partial a}{\partial r} & \ldots & * \\
		\vdots & \ddots & \vdots \\
		* & \ldots & \frac{\partial d}{\partial \theta} \\		
    \end{bmatrix}
\end{equation}
giving
\begin{equation}
    \det\left( J \right) = r^{3} \sin^{2}\left(\varphi_1\right) \sin\left(\varphi_2\right).
\end{equation}
The surface measure of $S^{3}$ is exactly
\begin{equation}
    \begin{split}
        \mu(S^{3}) &= \int^{2\pi}_{0} \int^{\pi}_{0} \int^{\pi}_{0}
        \sin^{2}\left(\varphi_1\right) \sin\left(\varphi_2\right)
        d\varphi_1 \ d\varphi_2 \ d\theta
        (?) \\
                   &= 2 \pi^{2}
    \end{split}
\end{equation}


\end{document}

