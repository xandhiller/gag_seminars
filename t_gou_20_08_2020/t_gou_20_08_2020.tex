\documentclass{article}
\usepackage{amsmath}      % Mathematics
\usepackage{amssymb}      % Mathematics
\usepackage{listings}     % Listings
%\usepackage{esint}       % Mathematics (Causing problems with mdframed)
\usepackage{color}        % Listings
\usepackage{courier}      % Listings
\usepackage[oldvoltagedirection]{circuitikz}   % Circuits
\usepackage{titlesec}     % Section Formatting
\usepackage{stmaryrd}     % \mapsfrom arrow. 
\usepackage{mathtools}    % \coloneqq
\usepackage{svg}
\usepackage{import}
\usepackage{pdfpages}
\usepackage{transparent}
\usepackage{xcolor}
\usepackage{blindtext}
\usepackage[hidelinks]{hyperref}
\usepackage{tabularx}
\usepackage{mdframed}
%%%%%%%%%%%%%%%%%%%%%%%%%%%%%%%%%%%%%%%%%%%%%%%%%%%%%%%%%%%%%%%%%%%%%%%%%%%%%%%%
% Math Macros
%%%%%%%%%%%%%%%%%%%%%%%%%%%%%%%%%%%%%%%%%%%%%%%%%%%%%%%%%%%%%%%%%%%%%%%%%%%%%%%%
% Standard Notation for Vectors in Computer Vision
\usepackage{mdframed}

% Footnote preferences
\renewcommand{\thefootnote}{\fnsymbol{footnote}}
  % Listings Prerequisites
%%%%%%%%%%%%%%%%%%%%%%%%%%%%%%%%%%%%%%%%%%%%%%%%%%%%%%%%%%%%%%%%%%%%%%%%%%%%%%%%
\definecolor{codegreen}{rgb}{0,0.6,0}
\definecolor{codegray}{rgb}{0.5,0.5,0.5}
\definecolor{codepurple}{rgb}{0.58,0,0.82}
\definecolor{backcolour}{rgb}{0.87,0.87,0.87}
\lstdefinestyle{mystyle}{
  backgroundcolor=\color{backcolour},   
  commentstyle=\color{codegreen},
  keywordstyle=\color{magenta},
  numberstyle=\tiny\color{codegray},
  stringstyle=\color{codepurple},
  basicstyle=\footnotesize\ttfamily,
  breakatwhitespace=false,         
  breaklines=true,                 
  captionpos=b,                    
  keepspaces=true,                 
  %numbers=left,                    
  numbersep=5pt,                  
  showspaces=false,                
  showstringspaces=false,
  showtabs=false,                  
  tabsize=2
}
\lstset{style=mystyle} 

% Allows you to refer to the author and title in text
\makeatletter
\let\inserttitle\@title
\makeatother
\makeatletter
\let\insertauthor\@author
\makeatother

%\newcommand{\qed}{$\square$}
\newcommand{\QED}{\hspace*{\fill}$\square$}
\newcommand{\iddots}{\reflectbox{$\ddots$}}
%----------------------%
% Counter for Custom Environments
%----------------------% 
%\rtask{label}
%\section{Task \ref{label}. Blah}
%----------------------%
% Custom Environments
%----------------------% 
\newcounter{question} 
\newcommand{\qanda}[2]{\AtEndDocument{\stepcounter{question}Q\arabic{question}. \textit{#1} \vspace{2mm}\newline \AtEndDocument{A\arabic{question}. #2 \newline \newline}}}
%----------------------%

/home/alex/Meta/templates/tex/dtper_article.tex
%%%%%%%%%%%%%%%%%%%%%%%%%%%%%%%%%%% FORMATTING %%%%%%%%%%%%%%%%%%%%%%%%%%%%%%%%%
\author{Tim Gou}
\title{\textbf{Groups, Analysis, and Geometry Seminars}: Harmonic Analysis of $\mathcal{SU}(2)$}
\date{20th August, 2020}
\usepackage{geometry} % Formatting
%%%%%%%%%%%%%%%%%%%%%%%%%%%%%%%%%%%%%%%%%%%%%%%%%%%%%%%%%%%%%%%%%%%%%%%%%%%%%%%%

\begin{document}

\maketitle

Suppose $f$ is $2\pi$-periodic, complex valued, integrable over $[0, 2\pi)$, then
\begin{equation}
    \widehat{f}(n) = \frac{1}{2\pi} \int^{2\pi}_{0} f(x) e^{-inx} \ dx
\end{equation}
with $n \in \mathbb{Z}$, is the Fourier transform of $f$. Question: What is $e^{inx}$? Why is $n \in \mathbb{Z}$? Answer: $e^{inx}$ is the character of circle group, denoted by $\mathbb{T}$ (i.e. $e^{ix} \in \mathbb{T}$).

A character is a continuous homomorphism from a locally compact Abelian group $G$ to $\mathbb{T}$: $ \chi : G \rightarrow \mathbb{T} $
where
\begin{equation}
    \chi(gh) = \chi(g)\chi(h)
\end{equation}
for $g,h \in G$. Let's work out $\chi : \mathbb{R} \rightarrow \mathbb{T}$ first, where $\mathbb{R}$ is the group $(\mathbb{R}, +)$. Since $\chi(0)=1$ (identity to identity) and $\chi$ is continuous, then $\exists a > 0$ such that
\begin{equation}
    \int^{a}_{0} \chi(y) \ dy.
\end{equation}
Let $\xi = \int^{a}_{0} \chi(y) \ dy$, then 
\begin{equation}
    \chi(x)\xi = \int^{a}_{0} \chi(x+y) \ dy = \int^{a+x}_{x} \chi(t) \ dt
\end{equation} 
so
\begin{equation}
    \chi(x) = \xi^{-1} \int^{a+x}_{x} \chi(t) \ dt
\end{equation}
and
\begin{equation}
    \begin{split}
        \chi'(x) &= \xi^{-1} \left(\chi(a+x) - \chi(x) \right) \\
                 &= \xi^{-1} \chi(x) \left(\chi(a) - 1\right) \\
                 &= c \chi(x).
    \end{split}
\end{equation}
We have an ODE
\begin{equation}
    \chi'(x) = c \chi(x)
\end{equation}
where solving the equation gives us
\begin{equation}
    \chi(x) = e^{cx}.
\end{equation}

\end{document}

